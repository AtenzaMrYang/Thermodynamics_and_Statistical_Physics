\section{热力学·统计物理2023年第二学期期末}

\subsection{填空题(每小题3分,共6小题,计18分)}
\begin{question}{题目1}
    1mol理想气体的物态方程为\underline{\qquad}\underline{\qquad}
\end{question}

\begin{question}{题目2}
    当系统在恒定的外界压强$p$下体积由$V_1$变为$V_2$时,外界对系统所做的功为$W=$\underline{\qquad}\underline{\qquad}
\end{question}

\begin{question}{题目3}
    在等温等容的条件下,系统的\underline{\qquad}\underline{\qquad}永不增加
\end{question}

\begin{question}{题目4}
    某系统有两个能级,第一个能级有三个量子态,第二个能级有两个量子态。系统由四个全同的玻色子组成,则两个粒子处于第一能级、两个粒子处于第二能级的微观状态数为\underline{\qquad}\underline{\qquad}
\end{question}

\begin{question}{题目5}
    由能量均分定理可知,温度为$T$时的$N$个弹性双原子分子组成的理想气体的内能是\underline{\qquad}\underline{\qquad}
\end{question}

\begin{question}{题目6}
    假设系统有20个全同粒子组成,粒子的自由度为3,则系统的自由度为\underline{\qquad}\underline{\qquad}
\end{question}


\subsection{简答题(每小题6分,共2小题,计12分)}
\begin{question}{题目1}
    对于等温等压系统,通常选择的特性函数是什么?为什么?
\end{question}

\begin{question}{题目2}
    正则系统适用于讨论粒子数、能量、体积一定的孤立系统。以上说法是否正确?为什么?
\end{question}

\subsection{推导证明题(每小题12分,共3小题,计36分)}
\begin{question}{题目1}
    能态方程给出了在温度保持不变时内能随体积的变化率与物态方程之间的关系,试推导之。
\end{question}

\begin{question}{题目2}
    试推导Fermi分布$\displaystyle a_l=\frac{\omega_l}{\mathrm{e}^{\alpha+\beta\varepsilon_l}+1}$
\end{question}

\begin{question}{题目3}
    试推导Bose统计中描述系统状态的参量$p$的统计表达式(即它与巨配分函数的关系)
\end{question}


\subsection{计算题(共3小题,计34分)}
\begin{question}{题目1}
    焦汤系数表示在焓不变的条件下,气体温度随压强的变化率。试求气体在节流膨胀过程中的焦汤系数。
\end{question}
\begin{question}{题目2}
    一维线性谐振子能量的经典表达式为
    $$
        \varepsilon=\frac{p^2}{2m}+\frac{1}{2}m\omega^2q^2
    $$
    试计算经典近似的谐振子配分函数及内能。
\end{question}
\begin{question}{题目3}
    试求绝对零度下自由电子气体中电子的平均速率。
\end{question}
