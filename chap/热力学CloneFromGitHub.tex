\section{热力学的基本规律}
\subsection{平衡态}
系统的各种宏观性质在长时间内不发生任何变化, 这样的状态称为热力学平衡态.

\subsection{什么样的系统是简单系统}
只需要体积和压强两个状态参量便可以确定系统的状态, 我们称这样的系统为简单系统.

\subsection{热平衡定律(第零定律)及意义}
\begin{itemize}
    \item \textbf{热平衡定律(第零定律)}: 如果两个物体各自与第三个物体达到热平衡, 它们彼此也必处在热平衡.
    \item \textbf{意义}: 根据热平衡定律可以证明处在平衡态下的系统态函数温度的存在. 热平衡定律不仅给出了温度的概念, 而且指明了比较温度的方法.
\end{itemize}

\subsection{三种系数}
体胀系数$\alpha$
$$
    \alpha=\frac{1}{V}\left(\frac{\partial V}{\partial T}\right)_p
$$
压强系数$\beta$
$$
    \beta=\frac{1}{p}\left(\frac{\partial p}{\partial T}\right)_V
$$
等温压缩系数$\kappa_T$
$$
    \kappa_T=\frac{1}{V}\left(\frac{\partial V}{\partial p}\right)_T
$$
\textbf{三种系数的关系}: 由于$p$, $V$, $T$三个变是之间存在函数关系$f(p,V,T)=0$(简单系统的物态方程), 其偏导数之间将存在下述关系
$$
    \left(\frac{\partial V}{\partial p}\right)_T\left(\frac{\partial
        p}{\partial T}\right)_V\left(\frac{\partial T}{\partial V}\right)_p=-1
$$
因此$\alpha$, $\beta$ , $\kappa_t$ 满足
$$
    \alpha=\kappa_t\beta p
$$

\subsection{物态方程}
简单系统的物态方程的一般形式
$$
    f(p,V,T)=0
$$
理想气体的物态方程
$$
    pV=nRT
$$
范氏方程
$$
    \left(p+\frac{an^2}{V^2}\right)(V-nb)=nRT
$$
\subsection{广延量和强度量}
\begin{itemize}
    \item \textbf{广延量}: 与系统的质量或物质的量成正比的热力学量.
    \item \textbf{强度量}: 与系统的质量或物质的量无关的热力学量.
\end{itemize}
\subsection{功的表达式}

\begin{enumerate}
    \item 在准静态过程\footnote{什么是准静态见课本P18}中, 外界对系统所作的功可以表示为
          $$
              \mathrm{d}\!\!\!^-W=-V\mathrm{d}V
          $$
          如果系统由$V_A$变到$V_B$, 则外界对系统所作的功等于上式积分
          $$
              W=-\int_{V_A}^{V_B}p\mathrm{d}V
          $$

    \item 磁化过程功的形式\footnote{推导见课本p23}:
          $$
              \mathrm{d}\!\!\!^-W=V\mathrm{d}\left(\frac{\mu_0\mathcal{H}^2}{2}\right)+\mu_0V\mathcal{H}\mathrm{d}\mathcal{M}
          $$

    \item 准静态过程中功的一般表达式
          $$
              \mathrm{d}\!\!\!^-W=\sum_iY_i\mathrm{d}y_i
          $$
\end{enumerate}


\subsection{热力学三定律}
\begin{enumerate}
    \item 热力学第一定律(能量定恒定律): 自然界一切物质都具有能量, 能量有各种不同的形式,
          可以从一种形式转化为另一形式, 从一个物体传递到另一个物体, 在传递与转化中能量
          的数量不变. 另一个表述形式: 第一类永动机是不可能造成的. 数学形式为:
          $$
              \mathrm{d}U=\mathrm{d}\!\!\!^-Q+\mathrm{d}\!\!\!^-W
          $$

    \item 热力学第二定律(两种表述形)
          \begin{itemize}
              \item 克氏表述: 不可能把热量从低温物体传到高温物体而不引起其它变化.
              \item 开氏表述: 不可能从单一热源吸热使之完全变成有用的功而不引起其它变化. (第二类永动机是不可能造成的)
          \end{itemize}
          数学表达形式:
          $$
              \oint\frac{\mathrm{d}\!\!\!^-Q}{T}\leqslant 0
          $$

    \item 热力学第三定律
\end{enumerate}

\subsection{热容量与焓}
等压过程中有:
$$
    C_p=\lim_{\Delta T\rightarrow 0} \left(\frac{\Delta Q}{\Delta T
    }\right)_p =\lim_{\Delta T\rightarrow 0} \left(\frac{\Delta U+p\Delta
        V }{\Delta T }\right)_p =\left(\frac{\partial U}{\partial
        T}\right)_p+p\left(\frac{\partial V}{\partial T}\right)_p
$$
引进态函数$H$, 名为焓:
$$
    H=U+pV \ \ \ \Delta H=\Delta U+p\Delta V
$$
则$C_p$可表示为
$$
    C_p=\left(\frac{\partial H}{\partial T}\right)_p
$$


\subsection{理想气体的定义}

\begin{itemize}
    \item 宏观: 内能只是温度的函数, 与体积无关的气体.
    \item 微观: 气体足够稀薄, 分子间的平均距离足够大, 相互作用能量可以忽略, 内能就与体积无关.
\end{itemize}
对于理想气体有
$$
    C_V=\frac{\mathrm{d}U}{\mathrm{d}T} \ \
    C_p=\frac{\mathrm{d}H}{\mathrm{d}T}
$$
并且可以证明(证明过程见课本P31): $C_p-C_V=nR$

\subsection{理想气体的四种过程}
%以理想气体的卡诺循环来说明理想气体的四种过程
\begin{itemize}
    \item 等温膨胀过程: 气体从状态$(p_1,V_1,T_1)$等温膨胀而达到状态$(p_2,V_2,T_1)$, 在这过程中外界做功
          $$
              W=-\int_{V_1}^{V_2}p\mathrm{d}V=-C\int_{V_1}^{V_2}\frac{\mathrm{d}V}{V}=-RT\ln\frac{V_2}{B_1}
          $$
          由于理想气体的\textbf{等温}膨胀过程中内能不变, 所以气体从外办吸收的热量为
          $$
              Q=-W=RT\ln\frac{V_2}{B_1}
          $$

    \item 绝热膨胀过程: 气体从状态$(p_1,V_1,T)$等温膨胀而达到状态$(p_2,V_2,T)$, 在这过程中外界做功
          $$
              W=-\int_{V_1}^{V_2}p\mathrm{d}V=-C\int_{V_1}^{V_2}\frac{\mathrm{d}V}{V^r}=\frac{C}{r-1}\left(\frac{1}{V_2^{r-1}}-\frac{1}{V_1^{r-1}}\right)
          $$
          但$p_1V_1^r=p_2V_2^r=C$,所以上式可以化为
          $$
              W=\frac{p_2V_2-p_1V_1}{r-1}=\frac{R(T_2-T_1)}{r-1}=C_V(T_2-T_1)
          $$
          由于是\textbf{绝热}过程, 故从外界收热$Q=0$.

    \item 等温压缩过程: 与等温膨胀过程类似.
    \item 绝热压缩过程: 与绝热膨胀过程类似.
\end{itemize}

\subsection{克劳修斯等式和不等式}
一个系统在一个循环过程中分别从$T_1$和$T_2$热源吸收的热量分别为$Q_1$和$Q_2$, 则有以下关系
$$
    \frac{Q_1}{T_1}+\frac{Q_2}{T_2}\leqslant 0
$$
上述称之为克劳修斯等式和不等式. 可推广为$n$个热源的情形: $\sum_{i=1}^n\frac{Q_i}{T_i}\leqslant
    0 $

\subsection{熵}
$A$和$B$是系统的两个平衡态. 则熵由下式定义
$$
    S_B-S_A=\int_A^B\frac{\mathrm{d}\!\!\!^-Q}{T}
$$
微分形式
$$
    \mathrm{d}S=\frac{\mathrm{d}\!\!\!^-Q}{T}
$$
由$\mathrm{d}U=\mathrm{d}\!\!\!^-Q+\mathrm{d}\!\!\!^-W$,
在可逆过程中, 如果只有体积变化功, 则有$\mathrm{d}\!\!\!^-W=-p\mathrm{d}V$所以有
$$
    \mathrm{d}S=\frac{\mathrm{d}U+p\mathrm{d}V}{T} \ \ \mathrm{或} \ \
    \mathrm{d}U=T\mathrm{d}S-p\mathrm{d}V
$$
\begin{itemize}
    \item 理想气体熵变: 对于$n$mol理想气体, 熵可表示为
          $$
              S=nC_{V,m}\ln T+nR\ln V+S_0 \ \ \mathrm{或} \ \ S=nC_{p,m}\ln
              T+nR\ln p+S_0'
          $$
          其中$S_0=n(S_{m0}-R\ln n)$,注意$S_0\neq S_0'$

    \item 熵增加原理: 系统经一过程由初态$A$变为终态$B$, 则有(推导见课本P55)
          $$
              S_B-S_A=\int_A^B\frac{\mathrm{d}\!\!\!^-Q}{T}\Rightarrow
              S_B-S_A\geq 0
          $$
          上式指出, 经过绝热过程后, 系统的熵永不减少, 等号适用于可逆过程, 不等号适用于不可逆过程. 因此系统在绝热条件下熵减少的过程是不可能实现的. 这个结论称为熵增加原理.
\end{itemize}

\subsection{自由能和吉布斯函数}
系统由等温过程由初态$A$到达终态$B$, 两态熵满足: $S_B-S_A\leqslant\frac{Q}{T}$. 又根据热力学第一定律, $U_B-U_A=Q+W$有
$$
    S_B-S_A\leqslant\frac{U_B-U_A-W}{T}
$$
为简化上式, 我们引进一个新的态函数:
$$F=U-TS$$
称为自由能, 则前式可简化为
$$
    F_A-F_B\leqslant-W
$$
上式表明, 在等温过程中, 系统对外界所作的功$-W$不大于其自由能的减少.
换句话说, 系统自由能的减少是在等温过程中从系统所能获得的最大功.
这个结论称为最大功定理.

%=================================第二章===========================================
\section{均匀物质的热力学性质}
\subsection{四个特性函数的全微分}
\begin{itemize}
    \item 热力学的基本方程, 即内能的全微: $\mathrm{d}U=T\mathrm{d}S-p\mathrm{d}V$
    \item 由焓的定义$H=U+pV$, 可得焓的全微分: $\mathrm{d}H=T\mathrm{d}S+V\mathrm{d}p$
    \item 由自由能定义$F=U-TS$, 可得自由能的全微分: $\mathrm{d}F=-S\mathrm{T}-p\mathrm{d}V$
    \item 由吉布斯函数定义$G=U-Ts+pV$, 可得其全微分: $\mathrm{d}G=-S\mathrm{d}T+V\mathrm{d}p$
\end{itemize}

\subsection{麦氏关系}
麦氏关系的四个表达式:
$$
    \left(\frac{\partial T}{\partial V}\right)_S=-\left(\frac{\partial p
    }{\partial S}\right)_V, \ \ \ \left(\frac{\partial T}{\partial
        p}\right)_S=-\left(\frac{\partial V }{\partial S}\right)_p
$$
$$
    \left(\frac{\partial S}{\partial V}\right)_T=-\left(\frac{\partial p
    }{\partial T}\right)_V, \ \ \ \left(\frac{\partial S}{\partial
        p}\right)_T=-\left(\frac{\partial V }{\partial T}\right)_p
$$

\subsection{气体的节流过程和绝热膨胀过程}
p77

\subsection{能态函数}
\subsection{平衡辐射场与温度体积的关系}
将空窖辐射看作热力学系统, 选温度$T$与体积$V$为状态参量, 空窖辐射的内能$U(T,V)$可以表示为
$$
    U(T,V)=u(T)V
$$
利用热力学公式
$$
    \left(\frac{\partial U}{\partial V}\right)_T=T\left(\frac{\partial p
    }{\partial T}\right)_V-p
$$
可得
$$
    u=\frac{T}{3}\frac{\mathrm{d}u}{\mathrm{d}T}-\frac{u}{3} \ \
    \mathrm{即} \ \ T\frac{\mathrm{d}u}{\mathrm{d}T}=4u
$$
积分得
$$
    u=aT^4
$$
其中$a$是积分常数, 上式指出空窖辐射的能量密度与绝对温度$T$的四次方成正比.

现求空窖辐射的熵, 将上式的$u$和$p=\frac{1}{3}u$(由列别节夫的实验可证明得到)代入热力学基本方程$\mathrm{d}S=\frac{\mathrm{d}U+p\mathrm{d}V}{T}$可有
$$
    \mathrm{d}S=\frac{1}{T}\mathrm{d}(aT^4V)+\frac{1}{3}aT^3\mathrm{d}V=4aT^2V\mathrm{d}T+\frac{4}{3}aT^3\mathrm{d}V=\frac{4}{3}a\mathrm{d}(VT^3)
$$
积分可得
$$
    S=\frac{4}{3}aT^3V
$$
积分中没有常数, 因为$V=0$时就不存在辐射场了. 在可逆绝热过程中辐射场的熵不变, 这时有
$$
    T^3V=\mathrm{常量}
$$

\subsection{磁介质系统的热力学基本方程}
当热力学系统中只包括介质而不包括磁场时, 功的表达式为
$$
    \mathrm{d}\!\!\!^-W=\mu_0\mathcal{H}\mathrm{d}\mathit{m}
$$
其中$\mathit{m}=\mathcal{M}V$是介质的总磁矩, 如果忽略磁介质的体积变化, 磁介质的热力学基本议程为
$$
    \mathrm{d}U=T\mathrm{d}S+\mu_0\mathcal{H}\mathrm{d}\mathit{m}
$$
由吉布斯函数$G=U-TS-\mu_0\mathcal{H}\mathit{m}$, 微分可得
$$
    \mathrm{d}G=-S\mathrm{d}T-\mu_0\mathit{m}\mathrm{d}\mathcal{H}
$$

\subsection{绝热去磁效应}
在绝热条件下减小磁场时, 磁介质的温度将降低, 这个效应称为绝热去磁致冷效应. 理论证明见P93-p94.

\section{单元系的相变}
\subsection{热动平衡判据}
\begin{itemize}
    \item 孤立系统在稳定平衡状态的必要和充分条件为:
          $$\Delta S<0$$
          泰勒展开, 准确到二级: $\Delta S=\delta
              S+\frac{1}{2}\delta^2S$, 当熵函数的一级微分$\delta
              S=0$, 熵有极值;当熵函数的一级微分$\delta
              S=0$, 二级微分$\delta^2S<0$, 熵有极大值. 由$\delta
              S=0$可以得到平衡条件, 由$\delta^2S<0$可以得到平衡的稳定性条件.

    \item 等温等容的系统处在稳定平衡状态的必要和充分条件为
          $$
              \Delta F<0
          $$
          对于$\Delta F=\delta F+\frac{1}{2}\delta^2F$, 由$\delta
              F=0$和$\delta^2F>0$可以确定平衡条件和平衡的稳定性条件.
    \item 等温等压的系统处在稳定平衡状态的必要和充分条件为
          $$
              \Delta G>0
          $$
          对于$\Delta G=\delta G+\frac{1}{2}\delta^2G$, 由$\delta
              G=0$和$\delta^2G>0$可以确定平衡条件和平衡的稳定性条件.
\end{itemize}

\subsection{吉布斯函数与摩尔G函数的关系}
由于吉布斯函数是广延量, 系统的吉布斯函数等于物质量$n$与摩尔吉布斯函数$G_m(T,p)$之积:
$$
    G(T,p,n)=nG_m(T,p)
$$

\subsection{化学势和巨热力势}
$\mu$称为化学势, 它等于在温度和压强保持不变的条件下, 增加1mol物质时吉布斯函数的改变. 化学势等于摩尔吉布斯函数
$$
    \mu=\left(\frac{\partial G}{\partial n}\right)_{T,p}=G_m
$$
定义$J$为巨热力势:
$$J=F-\mu n \ \ \mathrm{也可表示为: }  J=F-G=-pV$$
它的全微分为
$$
    \mathrm{d}J=-S\mathrm{d}T-p\mathrm{d}V-n\mathrm{d}\mu
$$

\subsection{单元系的复相平衡条件}
考虑一个单元两相系, 构成一个孤立系统. 用$\alpha$和$\beta$表示两相, 它的总内能, 总体积和总物质的量应是恒定的, 即
$$
    U^\alpha+U^\beta=\mathrm{常量}; \ \
    V^\alpha+V^\beta=\mathrm{常量}; \ \
    n^\alpha+n^\beta=\mathrm{常量}
$$
设想一虚变动, $\alpha$和$\beta$相的内能, 体积和物质的量分别发生虚变动
$\delta U^\alpha $, $\delta V^\alpha$, $\delta n^\alpha$ 和
$\delta U^\beta$, $\delta V^\beta$, $\delta
    n^\beta$, 孤立系条件要求
$$
    \delta U^\alpha+\delta U^\beta=0; \ \ \delta V^\alpha+\delta
    V^\beta=0; \ \ \delta n^\alpha+\delta n^\beta=0 \ \
$$
可得两相的熵变\footnote{对于开系有$\mathrm{d}G=-S\mathrm{d}T+V\mathrm{d}p+\mu\mathrm{d}n
    $, 见课本P108}分别为
$$
    \delta S^\alpha=\frac{\delta U^\alpha+p^\alpha\delta
        V^\alpha-\mu^\alpha\delta n^\alpha}{T^\alpha}=0; \ \ \delta
    S^\beta=\frac{\delta U^\beta+p^\beta\delta V^\beta-\mu^\beta\delta
        n^\beta}{T^\beta}=0;
$$
根据熵的广延性质, 整个系统的熵变是
$$
    \delta S=\delta S^\alpha+\delta S^\beta=\delta
    U^\alpha\left(\frac{1}{T^\alpha}-\frac{1}{T^\beta}\right)+\delta
    V^\alpha
    \left(\frac{p^\alpha}{T^\alpha}-\frac{p^\beta}{T^\beta}\right)-\delta
    n^\alpha\left(\frac{\mu^\alpha}{T^\alpha}-\frac{\mu^\beta}{T^\beta}\right)
$$
整个系统达到平衡时, 总熵有极大值, 必有
$$
    \delta S=0
$$
因为整个系统的熵变中$\delta U^\alpha$, $\delta V^\alpha$, $\delta
    n^\alpha$是可以独立改变的, $\delta S=0$要求
$$
    \frac{1}{T^\alpha}-\frac{1}{T^\beta}=0; \ \
    \frac{p^\alpha}{T^\alpha}-\frac{p^\beta}{T^\beta}=0; \ \
    \frac{\mu^\alpha}{T^\alpha}-\frac{\mu^\beta}{T^\beta}=0;
$$
即
$$
    T^\alpha=T^\beta; \ \ p^\alpha=p^\beta; \ \ \mu^\alpha=\mu^\beta;
$$
上式指出, 整个系统达到平衡时, 两相的温度, 压强和化学势必须分别相等. 这就是单元复相系达到平衡所要满足的平衡条件.

\subsection{克拉珀龙方程}
单元系三相共享时, 三相的温度, 压强和化学势势都必须相等, 即
$$
    T^\alpha=T^\beta=T^\gamma=T
$$
$$
    p^\alpha=p^\beta=p^\gamma=p
$$
$$
    \mu^\alpha(T,p)=\mu^\beta(T,p)=\mu^\gamma(T,p)
$$
设$(T,P)$和$(T+\mathrm{d}T,p+\mathrm{d}p)$是两相平衡曲线上邻近的两点. 在这两点上, 两相的化学势都相等:
$$
    \mu^\alpha(T,p)=\mu^\beta(T,p)
$$
$$
    \mu^\alpha(T+\mathrm{d}T,p+\mathrm{d}p)=\mu^\beta(T+\mathrm{d}T,p+\mathrm{d}p)
$$
两式相减, 得
$$
    \mathrm{d}\mu^\alpha=\mathrm{d}\mu^\beta
$$
上式表示, 当沿着平衡曲线由$(T,p)$变到$(T+\mathrm{d}T,p+\mathrm{d}p)$时, 两相的化学势的变化相等, 化学势的全微分为
$\mathrm{d}\mu=-S_m\mathrm{d}T+V_m\mathrm{d}p$.
其中$S_m$和$V_m$分别是摩尔熵和摩尔体积, 代入上式得
$$
    -S_m^\alpha\mathrm{d}T+V_m^\alpha\mathrm{d}p=-S_m^\beta\mathrm{d}T+V_m^\beta\mathrm{d}p
$$
或
$$
    \frac{\mathrm{d}p}{\mathrm{d}T}=\frac{S^\beta_m-S^\alpha_m}{V^\beta_m-V^\alpha_m}
$$
以$L$表示1mol物质由$\alpha$相转到$\beta$相时所吸收的相变潜热, 因为相变时物质的温度不变, 有$L=T(S_m^\beta-S_m^\alpha)$, 代入上式, 得
$$
    \frac{\mathrm{d}p}{\mathrm{d}T}=\frac{L}{T(V_m^\beta-V_m^\alpha)}
$$
上式称为克拉珀龙方程, 它给出了两相平衡曲线的斜率.

%==================五, 临界点的

\subsection{液滴的形成}
设液滴为$\alpha$相, 蒸气为$\beta$相, 表面为$\gamma$相. 三相的热力学基本方程分别为
$$
    \mathrm{d}U^\alpha=T^\alpha\mathrm{d}S^\alpha-p^\alpha\mathrm{d}V^\alpha+\mu^\alpha\mathrm{d}n^\alpha
$$
$$
    \mathrm{d}U^\beta=T^\beta\mathrm{d}S^\beta-p^\beta\mathrm{d}V^\beta+\mu^\beta\mathrm{d}n^\beta
$$
$$
    \mathrm{d}U^\gamma=T^\gamma\mathrm{d}S^\gamma+\sigma\mathrm{d}A
$$

热力学中, 把表面理想为几何面, 因此表面相的物质的量$n^\gamma=0$. 系统热平衡条件:
$$
    T^\alpha=T^\beta=T^\gamma
$$
设想在温度和总体积不变的条件下, 系统发生一个虑变动. 三相的物质的量, 体积和面积分别有$\delta
    n^\alpha$, $\delta V^\alpha$;$\delta n^\beta$, $\delta
    V^\beta$;$\delta
    A$的变化, 由于在虑变动中系统的总物质的量和总体积保持不变, 因此有
$$
    \delta n^\alpha+\delta n^\beta=0; \ \ \ \delta V^\alpha+\delta
    V^\beta=0
$$
在这虚变动中, 三相自由能的变化为
$$
    \delta F^\alpha=-p^\alpha\delta V^\alpha+\mu^\alpha\delta n^\alpha
$$
$$
    \delta F^\beta=-p^\beta\delta V^\beta+\mu^\beta\delta n^\beta
$$
$$
    \delta F^\gamma=\sigma\delta A
$$
在三相温度相等的条件下, 整个系统的自由能是三相的自由能之和, 整个系统自由能变化
$$
    \delta F=\delta F^\alpha+\delta F^\beta+\delta
    F^\gamma=-(p^\alpha-p^\beta)\delta V^\alpha+\sigma\delta
    A+(\mu^\alpha-\mu^\beta)\delta n^\alpha
$$

如果假定液滴是球形的, 有

$$
    V^\alpha=\frac{4\pi}{3}r^3, \ \ A=4\pi r^2
$$
$$
    \delta V^\alpha=4\pi r^2\delta r, \ \ \delta A^\alpha=8\pi r\delta
    r
$$
则整个系统自由能变化可化为
$$
    \delta F=-\left(p^\alpha-p^\beta-\frac{2\sigma}{r}\right)\delta
    V^\alpha+(\mu^\alpha-\mu^\beta)\delta n^\alpha
$$
根据自由能判据, 在温度和总体积不变的条件下, 平衡态的自由能最小, 必有$\delta
    F=0$. 因为$\delta V^\alpha$和$\delta n^\alpha$是任意的, 所以有
$$
    p^\alpha=p^\beta+\frac{2\sigma}{r}, \ \ \ \mu^\alpha=\mu^\beta
$$
上式是力学平衡条件, 它指出, 由于表面张力有使液滴收缩的趋势, 液滴的压强必须大于蒸气的压强才能维持力学平衡.

\subsection{液滴半径的讨论P127}
在一定的蒸气压强下$p'$下, 与蒸气达到平衡的液滴半径$r_c$为
$$
    r_c=\frac{2\sigma v^\alpha}{RT\ln\frac{p'}{p}}
$$
$r_c$称为中肯半径. 由上式可以看出, 对于$r>r_c$的液滴, 有$\mu^\alpha<\mu^\beta$, 因而液滴将继续凝结而增大,
对于$r<r_c$的液滴, 有$\mu^\alpha>\mu^\beta$, 因而液滴将气化而消失.

\subsection{一级, 二级相变}
\begin{itemize}
    \item 一级相变: 在相变点两相的化学势连续, 但化学势的一级偏导数存在突变:
          $$
              \mu^{(1)}(T,p)=\mu^{(2)}(T,p)
          $$
          $$
              \frac{\partial\mu^{(1)}}{\partial
                  T}\neq\frac{\partial\mu^{(2)}}{\partial T}, \ \ \
              \frac{\partial\mu^{(1)}}{\partial
                  p}\neq\frac{\partial\mu^{(2)}}{\partial p}
          $$

    \item 二级相变: 相变点两相的化学势和化学势的一级偏导数连续, 但化学势的二级偏导数存在突变, 称为二级相变. 因为
          $$
              c_p=T\left(\frac{\partial s}{\partial
                  T}\right)_p=-T\frac{\partial^2\mu}{\partial T^2}
          $$
          $$
              \alpha=\frac{1}{v}\left(\frac{\partial v}{\partial
                  T}\right)_p=\frac{1}{v}\frac{\partial^2\mu}{\partial T\partial p}
          $$
          $$
              \kappa_T=-\frac{1}{v}\left(\frac{\partial v}{\partial
                  p}\right)_T=-\frac{1}{v}\frac{\partial^2\mu}{\partial p^2}
          $$
\end{itemize}

\section{多元系的复相平衡的化学平衡}

%一, 偏摩尔量4.1.8  P148  4.1.15  4.1.16
%二, 多元系的复相平衡条件 4.2.4 三, P151. 4.3.6
%八, 三定律的两种表达形式成立的条件 绝对熵的表达式P172
%  等温过程, 节流, 自由膨胀, 量子态数的计算, 相律肯定考  玻色子

\subsection{偏摩尔量}
\begin{itemize}
    \item 选$T,p,n1,\cdots,n_k$为状态参量, 系统的三个基本热力学函数体积, 内能, 熵分别为
          $$
              V=V(T,p,n1\cdots n_k); \ \ U=U(T,p,n1\cdots n_k); \ \
              S=S(T,p,n1\cdots n_k)
          $$
          如果保持系统的温度和压强不变, 令系统中各组份都增为$\lambda$倍, 则有
          $$
              V=V(T,p,\lambda n1\cdots \lambda n_k)=\lambda V(T,p,n1\cdots n_k)
          $$
          $$
              U=U(T,p,\lambda n1\cdots \lambda n_k)=\lambda U(T,p,n1\cdots n_k)
          $$
          $$
              S=S(T,p,\lambda n1\cdots \lambda n_k)=\lambda S(T,p,n1\cdots n_k)
          $$
          由欧勒定理可知:
          $$
              V=\sum_in_i\left(\frac{\partial V}{\partial n_i}\right)_{T,p,n_j}, \
              \ U=\sum_in_i\left(\frac{\partial U}{\partial n_i}\right)_{T,p,n_j},
              \ \ S=\sum_in_i\left(\frac{\partial S}{\partial n_i}\right)_{T,p,n_j}
          $$
          定义
          $$
              v_i=\left(\frac{\partial V}{\partial n_i}\right)_{T,p,n_j}, \ \
              u_i=\left(\frac{\partial U}{\partial n_i}\right)_{T,p,n_j}, \ \
              s_i=\left(\frac{\partial S}{\partial n_i}\right)_{T,p,n_j}
          $$
          $v_i,u_i,s_i$分别称为$i$组元的偏摩尔体积, 偏摩尔内能和偏摩尔熵. 它们的物理意义是, 在保持温度, 压强和其它组元物质的量不变的条件下, 增加1mol的$i$组元物质时, 系统的体积(内能, 熵)r
          增量.
    \item 同样, 对于广延量熵有
          $$
              G=\sum_in_i\left(\frac{\partial G}{\partial
                  n_i}\right)_{T,p,n_j}=\sum_in_i\mu_i
          $$
          其中$\mu_i$是$i$组元的偏摩尔吉布斯函数.
    \item 对于多元复相系, 每一个相各有其热力学函数和热力学基本方程, 如$\alpha$相有
          $$
              \mathrm{d}U^\alpha=T^\alpha\mathrm{d}S^\alpha-p^\alpha\mathrm{d}V^\alpha+\sum_i\mu_i^\alpha\mathrm{d}n_i^\alpha
          $$
    \item 根据体积, 内能, 熵和物质的量的广延性质, 整个复相系统的体积, 内能, 熵和$i$组元的物质的量为
          $$
              V=\sum_\alpha V^\alpha, \ \ U=\sum_\alpha U^\alpha
          $$
          $$
              S=\sum_\alpha S^\alpha, \ \ n_i=\sum_\alpha n_i^\alpha
          $$
\end{itemize}

\subsection{多元系的复相平衡条件}
设两相$\alpha$和$\beta$都含$k$个组元, 并设热平衡和力学平衡条件已满足, 则多元系的相变平衡条件为
$$
    \mu_i^\alpha=\mu_i^\beta \ \ \ (i=1,2,\cdots,k)
$$
它指出整个系统达到平衡时, 两相中各组元的化学势必须分别相等.

\subsection{吉布斯相律}
多元复相系有$\varphi$个相, 每个相有$k$个组元, 则总数为$(k+1)\varphi$个的强度量中可以独立改变的只有$f$个(证明见: 课本P150):
$$
    f=(k+1)\varphi-(k+2)(\varphi-1) \ \ \mathrm{即:} f=k+2-\varphi
$$
上式称为吉布斯相律. $f$称为多元复相系的自由度数, 是多元复相系可以独立改变的强度量变量的数.

\subsection{三定律的两种表达形式成立的条件p170}
通常认为, 能氏定理和绝对零度不能达到原理是热力学第三定律的两种表述:
\begin{enumerate}
    \item 能氏定理: 凝聚系的熵在等温过程中的改变随绝地温度趋于零, 即
          $$
              \lim_{T\rightarrow0}(\Delta S)_T=0
          $$
          其中$(\Delta S)_T$指在等温过程中熵的改变
    \item 绝对零度不能达到原理: 不可能使一个物体冷却到绝对温度的零度.
\end{enumerate}

\subsection{绝对熵的表达式P172}
以绝对零度为参考态, 熵$S(T,V)$可表为
$$
    S(T,V)=\int_0^T\frac{C_V}{T}\mathrm{d}T
$$


\section{近独立粒子的最概然分布}
\subsection{\texorpdfstring{$\mu$}{μ}空间及代表点和轨道}
为了形象地描述粒子的力学运动状态, 用$q_1,\cdots,q_r;$, $p_1,\cdots,p_r$共$2r$个变量为直角坐标, 构成一个$2r$维空间, 称为\textbf{$\mu$空间}.

粒子在某一时刻的力学运动状态($q_1,\cdots,q_r$;
$p_1,\cdots,p_r$)可以用$\mu$空间中的一点表示, 称为粒子力学运动状态的\textbf{代表点}.

当粒子的运动状态随时间改变时, 代表点相应地在$\mu$空间中移动, 描画出一条\textbf{轨道}.

\subsection{自由粒子和线性谐振子}

\begin{itemize}
    \item 自由粒子是不受力的作用而作自由运动的粒子. 不存在外场时, 理想气体的分子或金属的自由电子都可近似看作自由粒子.

          当粒子在三维空间运动时, 它的自由度为3. 粒子任一时刻位置可由$x$, $y$, $z$确定, 动量为
          $$
              p_x=m\cdot{x}, \ \ p_y=m\cdot{y}, \ \ p_z=m\cdot{z}
          $$
          自由粒子的能量是其本身的动能
          $$
              \varepsilon=\frac{1}{2m}(p_x^2+p_y^2+p_z^2)
          $$
    \item 对于自由度为1的一维线性谐振子, 在任一时刻, 粒子的位置由它的位移$x$确定, 动量为$p=m\cdot{x}$. 它的能量是其动能和势能之和:
          $$
              \varepsilon=\frac{p^2}{2m}+\frac{A}{2}x^2=\frac{p^2}{2m}+\frac{1}{2}m\omega^2x^2
          $$
          代表点在$\mu$空间中轨道是上式确定的一个椭圆, 其标准形式:
          $$
              \frac{p^2}{2m\varepsilon}+\frac{x^2}{\frac{2\varepsilon}{m\omega^2}}=1
          $$
\end{itemize}

\subsection{什么叫系统的微观运动状态-气体的力学运动状态}
\subsection{玻色子和费米子}
自然界中微观粒子可分为两类:
\begin{itemize}
    \item 玻色子: 自旋量子数为半整数的“基本”粒子. 如电子, $\mu$子, 质子, 中子等.
    \item 费米子: 自旋量子数为整数的“基本”粒子. 如光子, $\pi$介子等.
\end{itemize}

\subsection{等概率原理}

等概率原理: 对于处在平衡状态的孤立系统, 系统各个可能的微观状态出现的概率是相等的. 等概率原理是平衡态统计物理的基本假设.

\subsection{玻色和费米系统的与分布\texorpdfstring{\{$a_l$\}}{{al}}相应的微观状态数}
\begin{itemize}
    \item 玻色系统与分布$\{a_l\}$相应的微观状态数为
          $$
              \Omega_{B.F.}=\prod_l\frac{(\omega_l+a_l-1)!}{a_l!(\omega_l-1)!}
          $$
    \item 费米系统与分布$\{a_l\}$相应的微观状态数为
          $$
              \Omega_{F.D.}=\prod_l\frac{\omega_l!}{a_l!(\omega_l-a_l)!}
          $$
\end{itemize}

\subsection{三种分布的关系}

\begin{itemize}
    \item 玻耳兹曼分布
          $$
              a_l=\omega_le^{-\alpha-\beta\varepsilon_l}
          $$
    \item 玻色分布
          $$
              a_l=\frac{\omega_l}{e^{\alpha+\beta\varepsilon_l}-1}
          $$
    \item 费米分布
          $$
              a_l=\frac{\omega_l}{e^{\alpha+\beta\varepsilon_l}+1}
          $$
\end{itemize}

其中参数$\alpha$和$\beta$由下述条件确定:
$$
    \sum_la_l=N, \ \ \ \sum_l\varepsilon_la_l=E
$$

\textbf{三种分布的关系:}
如果参数$\alpha$满足条件: $e^\alpha\gg1$时, 玻色和费米分布中分母中的$\pm1$就可以忽略, 这时玻色分布式和费米分布式都过渡到玻耳兹曼分布.

当以下关系
$$\frac{a_l}{\omega_l}\ll1 \ \ \ (\mathrm{对所有}l)$$
得到满足时, 有
$$
    \Omega_{B.E.}\approx\frac{\Omega_{M.B.}}{N!}\approx\Omega_{F.D.}
$$
