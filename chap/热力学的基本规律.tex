\section{热力学的基本规律}

\begin{question}{题目1.1}
    试求理想气体的体胀系数$\alpha$,压强系数$\beta$ 和等温压缩系数 $\kappa_T$.
\end{question}
\begin{solution}
    对理想气体状态方程两边取偏微分
    $$
        V\,\partial{p}+p\,\partial{V}=\nu R\,\partial{T}
    $$
    体胀系数$\alpha$
    $$
        \alpha=\frac{1}{V}\left(\frac{\partial V}{\partial T}\right)_p
        =\frac{1}{V}\frac{\nu R}{p}
        =\frac{1}{T}
    $$
    压强系数$\beta$
    $$
        \beta=\frac{1}{p}\left(\frac{\partial p}{\partial T}\right)_V
        =\frac{1}{p}\frac{\nu R}{V}
        =\frac{1}{T}
    $$
    等温压缩系数$\kappa_T$
    $$
        \kappa_T=-\frac{1}{V}\left(\frac{\partial V}{\partial p}\right)_T
        =-\frac{1}{V}\left(-\frac{V}{p}\right)
        =\frac{1}{p}
    $$
\end{solution}



\begin{question}{题目1.2}
    试证明:对于任意一种具有两个独立参量$T,p$的物质,如果通过实验测得其体胀系数$\alpha$及等温压缩系数$\kappa_T$,那么它的物态方程必定可以通过下述积分求得
    \begin{equation}\label{物态方程的积分形式}
        \ln{V}=\int(\alpha\,\mathrm{d}T-\kappa_T\,\mathrm{d}p)
    \end{equation}
    如果$\alpha=\dfrac{1}{T}$,$\kappa_T=\dfrac{1}{p}$,试求物态方程.
\end{question}
\begin{solution}
    (1) 根据体胀系数$\alpha$和等温压缩系数$\kappa_T$的定义,积分化为
    $$
        \ln{V}=\int\frac{1}{V}\left(\frac{\partial V}{\partial T}\right)_p\mathrm{d}T-\left(-\frac{1}{V}\right)\left(\frac{\partial V}{\partial p}\right)_p\mathrm{d}p
        =\int\frac{1}{V}\left[\left(\frac{\partial V}{\partial T}\right)_p\mathrm{d}T+\left(\frac{\partial V}{\partial p}\right)_p\mathrm{d}p\right]
    $$
    由于$T,p$两个参量相互独立,所以中括号内的表达式恰好是$V(T,p)$的全微分
    $$
        \ln{V}=\int\frac{1}{V}\,\mathrm{d}V\equiv\ln{V}
    $$
    (2) 如果我们代入已知条件$\alpha=\cfrac{1}{T}$,$T=\cfrac{1}{p}$
    $$
        \ln{V}
        =\int\left(\frac{1}{T}\,\mathrm{d}T-\frac{1}{p}\,\mathrm{d}p\right)
        =\ln{T}-\ln{p}+C
    $$
    化简得到
    $$
        pV=CT \quad (C>0)
    $$
\end{solution}


\begin{question}{题目1.4}
    在0℃和1atm下,测得一铜块的体胀系数和等温压缩系数分别为$\alpha=4.85\times10^{-5}\,\mathrm{K^{-1}}$和$\kappa_T=7.8\times10^{-7}\rm{atm^{-1}}$. $\alpha$和$\kappa_T$可近似看作常量. 现将铜块加热至10℃. 问:
    \begin{enumerate}
        \item 压强要增加到多少才能使铜块的体积维持不变?
        \item 若压强增加到100atm,铜块的体积改变多少?
    \end{enumerate}
\end{question}
\begin{solution}
    根据此前已证明的物态方程$\eqref{物态方程的积分形式}$
    \begin{equation}\label{物态方程的微分形式}
        \frac{1}{V}\,\mathrm{d}V=\alpha\,\mathrm{d}T-\kappa_T\,\mathrm{d}p
    \end{equation}
    (1) 加压能够完全抑制铜块的体积变化,所以$\mathrm{d}V=0$
    $$
        \Delta{p}
        =\frac{\alpha}{\kappa_T}\Delta{T}
        =\frac{4.85\times 10^{-5}}{7.8\times 10^{-7}}\times 10
        =622 {\,\rm Pa}
    $$
    (2) 铜块的体积随压强和温度变化,所以
    $$
        \frac{\Delta{V}}{V}
        =\alpha\Delta{T}-\kappa_T\Delta{p}
        =4.85 \times 10^{-5} \times 10 - 7.8 \times 10^{-7} \times 99
        =4.07 \times 10^{-4}
    $$
\end{solution}


\begin{question}{题目1.7}
    抽成真空的小匣带有活门,打开活门让气体冲入. 当压强达到外界压强$p_0$时将活门关上. 试证明:小匣内的空气在没有与外界交换热量之前,它的内能$U$与原来在大气中的内能$U_0$之差为$U-U_0=p_0V_0$,其中$V_0$是它原来在大气中的体积. 若气体是理想气体,求它的温度和体积.
\end{question}
\begin{solution}
    (1) 根据热力学第二定律
    $$
        U-U_0=W+Q
    $$
    由于过程迅速,可以认为系统与外界没有热量交换,即
    $$
        Q=0
    $$
    且只有外界对系统等压做功,而小匣对外界涌入的气体无阻碍作用
    $$
        W=p_0V_0
    $$
    所以
    $$
        U-U_0=p_0V_0
    $$
    (2) 如果气体是理想气体,则满足理想气体状态方程
    $$
        p_0V_0=nRT_0
    $$
    和
    $$
        C_V=\frac{nR}{\gamma-1}, \quad C_p=\gamma\frac{nR}{\gamma-1}, \quad U=\int_{}^{}C_V\,\mathrm{d}T
    $$
    所以
    $$
        U-U_0
    $$
\end{solution}


\begin{question}{题目1.8}
    满足$pV^n=C$(常量)的过程称为多方过程,其中常数$n$称为多方指数. 试证明:理想气体在多方过程中的热容$C_n=\cfrac{n-\gamma}{n-1}C_V$
\end{question}
\begin{solution}
    多方过程的热容为
    $$
        C_n=\lim_{\Delta{T}\to0}\left(\frac{\Delta Q}{\Delta T}\right)_n
        =\lim_{\Delta{T}\to0}\left(\frac{\Delta U+p\Delta V}{\Delta T}\right)_n
        = \left(\frac{\partial U}{\partial T}\right)_n+p\left(\frac{\partial V}{\partial T}\right)_n
    $$
    理想气体的内能只与温度有关
    $$
        C_V = \left(\frac{\partial U}{\partial T} \right)_n
    $$
    再对多方过程方程两边取全微分
    $$
        V^{n-1}\,\mathrm{d}T+(n-1)V^{n-2}T\,\mathrm{d}V=0
    $$
    结合理想气体状态方程得到
    $$
        \left(\frac{\partial V}{\partial T}\right)_n = \frac{V}{(n-1)T}
    $$
    最终
    $$
        C_n=C_V-\frac{pV}{T(n-1)}=\frac{n-\gamma}{n-1}C_V
    $$
\end{solution}


\begin{question}{题目1.9}
    假设气体的定压热容和定容热容是常量,试证明:若理想气体在某一过程中的热容$C_n$是常量,则该过程一定是多方过程,且多方指数为$n=\cfrac{C_n-C_p}{C_n-C_V}$
\end{question}
\begin{solution}
    根据热力学第一定律
    $$
        C_V\,\mathrm{d}T=C_n\,\mathrm{d}T-p\,\mathrm{d}V
    $$
    利用理想气体的物态方程消去$p$并结合$C_p-C_V=nR$
    $$
        (C_n-C_V)\frac{\mathrm{d}T}{T}=(C_p-C_V)\frac{\mathrm{d}V}{V}
    $$
    进一步,利用理想气体状态方程的微分形式
    $$
        \frac{\mathrm{d}p}{p}+\frac{\mathrm{d}V}{V}=\frac{\mathrm{d}T}{T}
    $$
    消去$T$
    $$
        (C_n-C_V)\frac{\mathrm{d}p}{p}-(C_n-C_p)\frac{\mathrm{d}V}{V}=0
    $$
\end{solution}


\begin{question}{题目1.16}
    理想气体分别经等压过程和等容过程,温度由$T_1$升至$T_2$. 假设$\gamma$是常数,试证明前者的熵增为后者的$\gamma$倍.
\end{question}
\begin{solution}
    等压过程的熵增为
    $$
        \Delta{S_1}=C_p\ln\frac{T_2}{T_2}
    $$
    等同过程的熵增为
    $$
        \Delta{S_2}=C_V\ln\frac{T_2}{T_2}
    $$
    所以
    $$
        \gamma=\frac{\Delta{S_1}}{\Delta{S_2}}=\frac{C_p}{C_V}
    $$
\end{solution}


\begin{question}{题目1.19}
    均匀杆的温度一端为$T_1$,另一端为$T_2$. 试计算达到均匀温度$\cfrac12(T_1+T_2)$后的熵增.
\end{question}
\begin{solution}
\end{solution}


\begin{question}{题目1.21}
    物体的初温$T_1$高于热源的温度$T_2$. 有一热机在此物体与热源之间工作,直到将物体的温度降低到$T_2$为止. 若热机从物体吸取的热量为$Q$,试根据熵增加原理证明:此热机所能输出的最大功为
    $$
        W_{\max}=Q-T_2(S_1-S_2)
    $$
    其中$S_1-S_2$是物体的熵变.
\end{question}
\begin{solution}
\end{solution}


\begin{question}{题目1.23}
    简单系统有两个独立参量. 如果以$T,S$为独立参量,可用纵坐标表示温度$T$,横坐标表示熵$S$,构成$T-S$图. 图中的一点与系统的一个平衡态相对应,一条曲线与一个可逆过程相对应. 试在图中画出可逆卡诺循环过程的曲线,并利用$T-S$图求卡诺循环的效率.
\end{question}
\begin{solution}
\end{solution}